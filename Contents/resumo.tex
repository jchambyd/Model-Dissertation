\keyword{ABNT}
\keyword{Processadores de Texto}
\keyword{Formatação eletrônica de documentos}
% Resumo na língua do documento.
\begin{abstract}
    Consiste na apresentação clara e concisa dos pontos relevantes do trabalho,
    de maneira a permitir ao leitor saber da conveniência ou não da sua leitura na íntegra.
    É redigido pelo autor, em português e em inglês, em páginas distintas.
    Cada resumo ocupará no máximo uma folha e terá até 500 palavras.
    Para maiores informações com relação à redação consultar a NBR 6028 da ABNT.
    Quanto ao estilo, o resumo deve ser composto por uma sequência de frases
    completas e não por uma enumeração de tópicos; a primeira frase deverá ser
    significativa, explicando o tema principal do documento.
    Na redação, dar preferência ao uso da terceira pessoa do singular e do verbo na voz ativa.
    Após o resumo e o abstract devem constar palavras-chave relativas aos assuntos
    da monografia, em português e inglês respectivamente, e separadas entre si por ponto.
\end{abstract}