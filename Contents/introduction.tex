% Início do texto.
\chapter{Introdução}

Este capítulo tem o objetivo de descrever os detalhes necessários à correta
formatação do documento.  As informações aqui apresentadas devem ser
suficientes para formatar corretamente o documento com qualquer ferramenta de
edição.  Este documento foi criado utilizando estilos.  Observe isso com
atenção.
% Ou use Latex e seja mais feliz.

Os capítulos são sempre iniciados em uma nova folha.  O título do capítulo é
formatado todo em letras maiúsculas, com fonte Times ou Arial, tamanho 12, em
negrito.  Para os numerados, é alinhado à esquerda, precedido do respectivo
número.  Em relação às margens, margem superior e esquerda será 3 cm, e margem
inferior e direita será 2 cm.  Os títulos das seções primárias devem começar na
parte superior da margem da folha e separada do texto que o sucede por dois
espaços de 1,5 cm.  O alinhamento do texto é justificado, e dos títulos das
seções o alinhamento é à esquerda.  Para títulos de seções que não são
numerados (resumo, abstract, sumário, referências, listas, apêndices, anexos,
etc.), o alinhamento é centralizado.

\section{Sobre os Títulos e Capítulos}

As demais subdivisões do texto (seções, subseções, etc.) são formatadas com o
título alinhado sempre à esquerda, precedido da respectiva numeração. Esta é
formada pela união dos números relativos a cada nível de subdivisão, separados
por pontos. Não se inclui um ponto no final.

São permitidas subdivisões até o 5\textsuperscript{o} nível (onde o capítulo é o 1\textsuperscript{o} nível). Os parâmetros para formatação dos títulos e espaçamentos nos diversos níveis de subdivisões são apresentados na \autoref{tab:tamanhos}:

\begin{table}[htbp]
	\centering
	\caption{Parâmetros para formatação das subdivisões do texto}
    \label{tab:tamanhos}
	\begin{tabular}{lllll} \toprule
		Nível        & Tamanho & Estilo           & Esp. Antes & Esp. Depois \\ \midrule
		1 (capítulo) & 12 pt   & negrito, maiúsc. & 0 pt       & 0 pt \\
		2 (seção)    & 12 pt   & negrito          & 0 pt       & 0 pt \\
		3 (subseção) & 12 pt   & negrito          & 0 pt       & 0 pt \\
		4            & 12 pt   & itálico          & 0 pt       & 0 pt \\
		5            & 12 pt   & normal           & 0 pt       & 0 pt \\ \bottomrule
	\end{tabular}
	\legend{Fonte: \citeonline[p. 49-56]{furaste}.}
\end{table}

\subsection{Sobre o Sumário}
Relaciona as principais divisões e seções do texto, na mesma ordem em que nele se
sucedem, indicando, ainda, as respectivas páginas iniciais. O sumário deverá ser localizado
antes da introdução. Para maiores detalhes, ver a Norma NBR 6027 da ABNT. Elemento
obrigatório onde se relacionam as principais seções do texto na mesma ordem e grafia em que
nele se sucedem, com a indicação da paginação inicial. Em monografias deve ser o último
elemento pré-textual e deve iniciar no anverso de uma folha.

Os indicativos das seções (números arábicos) que compõem o sumário devem ser
alinhados à esquerda. Recomenda-se que os títulos das seções e subseções sejam alinhados
pela margem do titulo indicativo mais extenso (como exemplo, ver o sumário desta
publicação). Todas as seções devem conter um texto relacionado a elas. A designação da
página das seções e subseções pode apresentar somente a página inicial ou a página inicial e final separadas por hífen.

\subsubsection{Sobre a Lista de Abreviaturas e Siglas}
Todas as abreviaturas e siglas devem ser ordenadas alfabeticamente e seguidas de seus
respectivos significados. Um exemplo pode ser visualizado no início deste documento.

\subsubsection{Sobre a Lista de Símbolos}
Semelhante à lista de abreviaturas e siglas, os símbolos utilizados no documento
devem ser apresentados na ordem em que nele aparecem acompanhados de seus respectivos
significados.

\subsubsection{Sobre as Listas de Figuras e de Tabelas}
Separadamente para as Figuras e Tabelas, devem ser relacionadas às ilustrações na
ordem em que aparecem no texto, indicando, para cada uma, o seu número, legenda e página
onde se encontra.

\section{Numeração das Páginas}
Os números de página são colocados na margem superior do documento, a 2 cm da
borda superior do papel, alinhados à margem externa do texto. Por margem externa entende-
se a margem direita nas páginas ímpares e a esquerda nas páginas pares. Quando o documento
é produzido somente anverso, utiliza-se sempre a margem direita para a numeração. Todas as
páginas do documento, a partir da folha de rosto, são contadas, mas a numeração só é
mostrada a partir do primeiro capítulo de texto propriamente dito, ou seja, normalmente a
Introdução. Assim, as primeiras páginas não devem apresentar numeração.